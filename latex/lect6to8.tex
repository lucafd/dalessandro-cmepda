\section{Lezione 6, lunedì 10 ottobre 2022}

\subsection{Testing and Documenting}

\subsubsection{Testing}

Non c'è modo di sapere che sia sempre corretto il programma, e quindi per sua natura sarà non corretto o ci sarà un caso in cui si rompe.
Quali sonon le tecniche per sapere all'incirca che il nostro programma sta facendo la cosa giusta?
Molto difficile per un linguaggio interpretato come python. Se compilato certe verifiche le fa il compilatore, mentre se interpretato non lo sa nessuno a priori.
Due attività:
\begin{itemize}
	\item analisi statica della sintassi, analisi del codice e di verifiche della sintassi. \textbf{Pylint}, usalo!
	\item \textbf{unit testing}, test dell'unità. Isolare funzionalità il più possibile elementari, e verificare che quel pezzettino faccia la cosa giusta. Piccolo test, verificato in una cosa relativamente piccola ed ho una metrica che lo controlla. L'esempio non è per dire che in ogni caso fa la cosa giusta, ma vanno isolati i casi interessanti. Spezzo il codice in parti elementari. Le funzioni più sono piccole e focalizzate più sono fatte meglio. Va realizzato con una scelta oculata di funzioni e classi.
\end{itemize}

\textbf{Variabili d'ambiente}: sono la chiave per il funzionamento del sistema operativo. Su internet si trova. Scrivere un backslash in modo che sia un backslash davvero? \mintinline{python}{\\}.

\textbf{Encoding}: modo di codificare i caratteri del sistema. Corrispondenza fra un glifo e un numero binario/esadecimale. Windows in automatico usa Latin1, non UTF-8. Errori di Unicode: codificare un file con un codice diverso da quello con cui è stato scritto.

\mintinline{python}{assert}, è una parola riservata, valuta un espressione, se è vera non fa nulla, se è falsa genera un \texttt{AssertingError}. In generale non va usato mai l'assert, perché in caso di errore non posso modificarlo in corsa. Sensata solo quando si è certi del risultato e usato nel debugging.

Per i test però non ha senso fare tanti piccoli test manuali, voglio trovare un modo generalissimo di farlo. All'aumentare della complessità del programma, capita che fai un cambiamento e questo cambiamento distrugge altre parti che interagiscono con il programma se fatto male.
Se cresce organicamente con una serie di unit tests, riesce a crescere bene e senza rompersi.
Una funzione deve incapsulare una funzione semplice e ben definita. Unit testing è assicurarsi che ogni pezzo faccia quello che deve fare.
Un paradigma particolare è il Test-Driven Development (TDD): prima di scrivere la funzione scrivo il test, poi scrivo il corpo vuoto della funzione, e alla fine implemento la funzione fino a quando il test non passa. Verifica sempre che si può fare prima il test!
Una cosa da convincere è che il test e la documentazione vanno fatti fin dall'inizio.
Forzarsi a scrivere la documentazione e il test.

Nell'esempio naïve delle slides, va sistemata la tipizzazione e i controlli su quello che viene passato. Vanno pensati i test per tutti i casi in cui viene usata.
Io non voglio fare nulla a mano però, completamente inutile. Voglio allora usare qualche framework per gli unittest, in particolare \texttt{unittest}, oppure \texttt{pytest}. Oggi tutti usano la seconda ma non è detto si a più efficace.
Test sono fatti in modo da poter runnare automaticamente. Nei test deve essere chiaro cosa voglio testare, e deve essere chiaro cosa voglio ottenere:
\begin{minted}{python}
import unittest

def square(x):
	"""Function returning the suare of x.
	In real life this would be in a differnt module!
	"""
	return x**2.

class TestSquare(unittest.TestCase):

	def test(self):
		"""Dumb unit test---make sure that the square of 2. is 4.
		"""
		self.assertAlmostEqual(square(2.), 4.) # usato almost equal per il discorso di float.

if __name__ == '__main__':
	unittest.main()	
\end{minted}

Se guardo un repo, ci sono due cartelle, \texttt{test} e \texttt{docs}. Ci sonon un certo numero di test. Il repo di esempio di github è `ixpeobssim` su github da lucabaldini.
Per fare fare un test, uno dovrebbe scrivere sul terminale \mintinline{python}{python -m unittest programName} e farglielo fare, oppure creare un \texttt{Makefile} piccolino che lo faccia in automatico.

Tipicamente uno di questi test non leggi mai l'output. Ma passa da solo. Non devo essere io che la triggero. A me interessa solo quando fallisce, ed è il concetto di continuous integration, ed è il concetto di \textit{Action} su github.

**TODO**: attaccare il repo ad un CI/CD.

Poi fa pylint.

Static typing/annotations, posso annotare le funzioni in modo che sappia che sono di un certo tipo. Però è per la leggibilità del codice, è solo annotazione.
\begin{minted}{python}
def square(x):
	"""Return the square of a number, but without annotations
	"""
	return x**2.
def annotated_square(x: float) -> float:
	"""Return the square of a number; this is the structure of an annotation.
	It just increases legibility.
	"""
	return x**2.	
\end{minted}

\subsubsection{Documenting}

Le pagine di documentazione non sono fatte a mano. Basta vedere il link al codice sorgente della funzione. 
Va notato che nella documentazione c'è il commento nella docstring.
Documentazione vive dentro il codice, e il meccanismo di documentazione sono le docstrings, fra apici tripli.
Non sono un posto dove le cose sono fatte a caso, ma permette a un tool automatico (\textbf{sphynx}) di generare automaticamente la documentazione.
Quali argomenti prende in ingresso, cosa fa, qualunque cosa in cui dobbiamo stare attenti.
How to use sphynx, learn it. \texttt{readthedocs} è open source.
Fare il tag di una nuova versione?
Automatizzare le cose, o lo sono automatiche o non succedono.
Per il progetto finale ce lo si aspetta.

\subsection{Numpy and SciPy}

Introdotto il concetto di applicabilità all'array elemento per elemento, e il concetto di broadcasting, cioè fra array di dimensioni opportune posso eseguire operazioni artmetiche.
\mintinline{python}{np.full((2,4), 3.)} fa una matrice $2\times4$ piena di soli tre.
In generale le operazioni le fa elemento per elemento, anche il prodotto. Però devono avere la stesssa dimensione.
Se hanno la stessa lunghezza, è ovvio moltiplicarlo, se con lunghezze diverse non è banale cosa significa.
Se usi broadcasting e ci si ragiona bene, fattore 100 sulla velocità.
\mintinline{python}{dir(object)} fornisce tutte le operazioni definite su quell'oggetto.

Quali sono le regole del broadcasting? Sulla documentazione di Scipy. Sono complesse.

\subsubsection{Maschere}

Supponiamo che vogliamo vedere dove convertono i fotoni nel silicio, la lunghezza di assorbimento.

\begin{minted}{python}
import random
import time
import numpy as np
import matplotlib.pyplot as plt

N = 1000000 # numero di fotoni
THICKNESS = 0.200 # mm, spessore del materiale del fotorilevatore

lambda_ = 0.100 # mm, e lambda in realtà è riservata e quindi uso l'underscore per poter usare quel nome

num_absorbed = 0.
abs_z = []

t0 = time.time()
# abbiamo notato l'errore dalla fisica e notiamo cosa fa expovariate, quindi vuole 1/lambda e non lambda
for i in range(N):
	z = random.expovariate(1. / lambda_)
	if z < THICKNESS:
		num_assorbed += 1
		abs_z.append(z)
elapsed_time = time.time() - t0
print(f'Running time: {elapsed_time}')

quantum_efficiency = num_absorbed / N
print(f'Quantum efficiency: {quantum_efficiency}')
# me la aspetto circa 1-e^-2, quindi circa 90%, non così poca.

plt.hist(abs_z, bins=100)
plt.yscale('log')
plt.show()
\end{minted}

Questo programma lavora linearmente con $N$, quindi non è banale il fatto che voglia risparmiare di tempo... 
Vettorizzare significa eliminare cicli for e sostituirlo con operazioni fra vettori. Quando possibile, questo fa la differenza come tempistiche.

\begin{minted}{python}
import random
import time
import numpy as np
import matplotlib.pyplot as plt

N = 1000000 # numero di fotoni
THICKNESS = 0.200 # mm, spessore del materiale del fotorilevatore

lambda_ = 0.100 # mm, e lambda in realtà è riservata e quindi uso l'underscore per poter usare quel nome
def eff_simple():
	abs_z = []
	t0 = time.time()
	for i in range(N):
		z = random.expovariate(1. / lambda_)
		if z < THICKNESS:
			num_assorbed += 1
			abs_z.append(z)
	elapsed_time = time.time() - t0
	print(f'Running time: {elapsed_time}')
	return abs_z

# ora vettorizziamo la funzione. Notiamo che numpy genera random in array, la numpy random.exponential
def eff_vectorized(num_events):
	# invece di generare un loop for, le genero tutte assieme, ma avviene in C e non in python che è lento.
	abs_z = np.random.exponential(lambda_, size=num_events)
	print(abs_z, len(z)) # questo solo di test per vedere che funziona.
	# questa si chiama maschera
	abs_z = abs_z[abs_z <= THICKNESS]
	elapsed_time = time.time() - t0
	print(f'Running time: {elapsed_time}')
	return abs_z

z = eff_simple(N)
quantum_efficiency = len(z) / N
print(f'Quantum efficiency: {quantum_efficiency}')
# me la aspetto circa 1-e^-2, quindi circa 90%, non così poca.
z = eff_vectorized(N)
quantum_efficiency = len(z) / N
print(f'Quantum efficiency: {quantum_efficiency}')
#
#plt.hist(abs_z, bins=100)
#plt.yscale('log')
#plt.show()
\end{minted}

Lo slicing invece cos'è? e Indexing, cioè come si fa in Numpy a prendere sottoinsiemi di array. Comunque sulle slides c'è.

Prima di tutto che cos'è una maschera? 

\begin{minted}{python}
import numpy as np

a = np.random.uniform(size=10)
mask = a > 0.5
\end{minted}

In questo esempio, il nuovo array mask creato è un array con soli true e false, un array di booleani dove è scritto in ogni cella se è vera o falsa la condizione scritta.
ma posso anche fare

\begin{minted}{python}
a[mask]
\end{minted}

e questo, passando una maschera con parentesi quadre, restituisce un array in cui gli elementi presenti corrispondono ai true della maschera.

Comunque si vede dal codice che è un risparmio notevole. Nella vita si fa questo, vettorizzo problemi che non si sanno fare altrimenti.
Per esempio se su un immagine ho cose sensate. In generale siamo abituati ai loop, ma non è efficiente.
Esempio di qualcosa di difficile da vettorizzare, nel mondo della fisica delle particelle.
Se faccio assorbire un fotone nel silicio di 5 keV, se 3.6, sono circa 1500 in media di coppie generate. Qual è la varianza su questo? Non è una poissoniana, perché c'è un fenomeno che non conosco. 
Siccome il numero non è fisso, questa è una cosa difficile da vettorizzare.

Libreria importante è \texttt{pandas}, che può fare comodo.
In generale leggere e scrivere file excel con python lo si fa con pandas.
\begin{minted}{python}
import pandas as pd
\end{minted}

Ora lavoriamo con l'\texttt{assignment\_basic\_4}

\begin{minted}{python}
import matplotlib.pyplot as plt
import numpy as np
import scipy.interpolate import InterpolatedUnivariateSpline

class ProbabilityDensityFunction(InterpolatedUnivariateSpline):
	# in teoria potrei anche commentare tutto e fare pass, perché in pratica sto rinominando la funzione con il programma base.
	def __init__(self, x, y):
		""" Construct
		"""
		super().__init__(x, y)

if __name__ == '__main__':
	x = np.linspace(0., np.pi, 20)
	y = np.sin(x)
	f = ProbabilityDensityFunction(x, y)

	print(f.integral(0., np.pi) # le spline hanno il metodo integral, e così posso normalizzare

	plt.plot(x, y, 'o')
	_x = np.linspace(0., np.pi, 200)
	plt.plot(_x, f(_x))

# parte da griglia, e interpola da griglia una funzione.
\end{minted}

Come si fa a generare un numero random? Primo metodo, hit or miss, e quindi da random uniforme creo coppie di punti e se $y$ è sotto $f(x)$ accetto x o meno.

Altrimenti, uso la funzione cumulativa e la ppf (percent-point function). La cumulativa mappa il punto nel quantile, se ruoto invece di 90 gradi, la nuova funzione mappa il quantile nel suo valore.
E se si prende un array random fra $0$ e $1$ e gli applico la ppf, avremo un array generato come la funzione di densità di probabilità. Si chiama inverse transform. L'abbiamo fatto a Analisi Statistica dei Dati. Esponenziale è un caso in cui la ppf si può generare analiticamente.
\textbf{TODO}: capire cos'è una spline. una funzione interpolatrice. Queste funzioni sono facilmente integrabili e derivabili. Facilmente normalizzabili, la cdf è gratis per l'integrale, e allora per generare numeri random vanno generati fra 0 e 1, ci applico la ppf e ho finito.

\texttt{splrand} sembra esserci più o meno un'implementazione davvero funzionante.

Questo è complesso come assegnamento. Prendere come ispirazione.

\section{Lezione 7, giovedì 13 ottobre 2022}

\subsection{Assegnamento 2}

Generare numeri random in due modi diversi, una con le spline functions, una per generare con l'inverso della fourier transform.
Ora guardo la funzione da \texttt{pdf.py} e voglio vedere come calcolare la cdf. Fa vedere che sto applicando gli argomenti giusti e calcolo correttamente la cdf. 
Unica cosa, non è detto che la lista che passo sia ordinata, quindi forse andrebbe specificato nella documentazione.
La cdf non è difficile da calcolare, invertirla è un casino, è il calcolo della ppf.
Se ho zone in cui prob è zero, allora ho problemi: la cdf sarà costante e allora nella spline gli passo valori tutti uguali, e per la spline è un casino.
Come filtro un array?

\begin{minted}{python}
a = np.array([1., 2., 2., 3., 4., 4.])
np.unique(a)
# questo ritorna solo i valori unici della y, 
np.unique(a, return_index = True)
\end{minted}

Concettualmente, il debugging che faccio a mano sul codice ha senso farlo nell'unittest.
Linea 69 del test, noti che se plotti con più punti, le discontinuità si fanno vedere! la spline non converge benissimo, perché di ordine 3...
Se invece di farla di ordine tre la faccio di ordine 1, interpoli, è molto più bello
In generale le funzioni discontinue vanno trattate in modo diverso... 
Negli strumenti che costruiscono gli astronomi, hanno i bordi perché sono oltre il $k_{edge}$ degli specchi in astronomia.

La normalizzazione va fatta a mano? cioè va fatta automaticamente.
fra la def di \texttt{init} e \texttt{super()}. Va rifatto a manina.

\begin{minted}{python}
norm = InterpolatedUnivariateSpline(x, y, k=k).integral(x[0], x[-1])
y /= norm
\end{minted}

```

\textbf{TODO} vedi a casa come si fa a vedere se l'array è ordinato.

\subsection{Errori ed altro}

\texttt{Traceback} e il tipo di errore.
Qua dice esattamente qual è il problema! Al contrario di git.
\texttt{lecture\_advanced\_2}
In \texttt{C} si ritorna codici di errore numerici, 0 tutto ok, numero diverso non ok.
In Python si usa altro, il meccanismo delle eccezioni (exceptions).
Nelle slides pagina tre c'è perché ha senso usare eccezioni e non flags.

\begin{minted}{python}
iterable_array[:5] # è lo slicing dell'array dalla posizione in 5 in poi
\end{minted}

Se guardo la funzione \mintinline{python}{cut_before} implementata a pagina 4, noto che dà un errore molto chiaro.
In python la filosofia di base è evitare di inventare. Ha senso. In generale posso inventarmi molte cose, molti modi di maneggiare, però fagli sputare l'errore direi che è la cosa migliore! Se la sottostringa non c'è, io gli ritorno il value error e lo sa che non sta passando.
Come intercetto però un errore e gli dico che voglio fare qualcosa di specifico?
Eccezioni.
Cosa sono? È un oggetto, che eredita da una classe di oggetti.
È \texttt{raised} quando lo segnalo che è andato male.
Posso intercettare (\texttt{caught}) l'eccezione, e posso dire che con il meccanismo di \texttt{TryExcept}:

\begin{minted}{python}
def cut_before(input_string, substring):
	try:
		result = input_string[:(input_string.index(substring))]
		print('This line is not executed if an exception is raised in the try block')
		return result
	# Catch the correct exception type with 'except'
	except ValueError:
		print('This line is executed only if a ValueError is raised in the try block')
\end{minted}

Se non si sollevano eccezioni, siamo nel primo branch, se invece incontro un eccezione, andiamo nel branch except corrispondente. 
Nota che l'except può essere eseguito in base al tipo di errore!
Qual è la logica? Non crasha!! E continuo a fare cose se so che tipo di errore esce sempre.
Si possono intercettare, e posso
Ok, \texttt{except:} si può fare ma preferirei non farlo e specifico il tipo di errore.
Tipico errore, try except \texttt{KeyError}, in generale dobbiamo intercettare l'errore nel modo più specifico possibile.
Cioè, un conto è se manca la chiave, un conto è se manca tutto il dizionario!
\texttt{else} e \texttt{finally} sono altri statement da usare.
Se le cose fossero solo così, non sono molto meglio. Però in generale il trick è che l'errore manda indietro tutto l'oggetto, l'\textit{eccezione}, e quindi passo tutte le informazioni che servono a ricostruire l'errore del file.
Molto generale, in cui ci sono gerarchie di eccezioni molto ricche che consentono di fare la gestione degli errori.
Pagina 10 c'è la gestione degli errori.
\texttt{with}. Se lo facciamo in modo brutale, solo con open, lascia le cose incasinate, ma posso usare with che è più sicuro.

\begin{minted}{python}
try:
	with open ('i_do_not_exist.txt') as lab_data_file:
	""" Do some process here...
	"""
	pass

except FileNotFoundError as e: 
	# we assign a name to the the exception, 
	# e la passiamo e maneggiamo come un oggetto!!
	print(e)

# We can be less specific by catching a parent exception
except OSError as e:
	# OSError is a parent class of FileNotFoundError
	print(e)

# catching Exception will catch almost everything!
except Exception as e:
	print(e)
\end{minted}

Generalmente, non dovremmo mai chiamare \texttt{Exception} perché è troppo generale.
Even worse, you should never catch for \texttt{BaseException} 
as that would even prevent the user from aborting the execution with a \texttt{KeyboardInterrupt}, ad esempio usando \texttt{Ctrl-C}.

Asking permission: prima guardo se c'è il file e poi lo apro. In Python come costo computazionale è il contrario, è più facile provare e otterere l'errore che il contrario! Differenza filosofica grande. In C è il contrario.

fare \texttt{if: else:} non conviene!

Esiste il \texttt{logging}, non lo ha guardato.

Possiamo sollevare eccezioni da noi!

Ad esempio pagina 20 delle slides, \texttt{raise RuntimeError}
Ricorda, in Python si può fare cose violente, tipo \texttt{sys.exit()}. Questa è un opzione, ma va valutata! Se metto l'uscita, non posso intervenire! Se invece uso l'errore, chi usa il codice e lo estende, ha l'opzione di estendere il codice e può intercettare e gestire l'eccezione. Sollevare l'opzione, lascia la scelta all'utente.

Possiamo creare un eccezione in maniera granulare addirittura, con una classe appostita che eredita da \texttt{(Exception)}.

Ad esempio, pagina 23 fa la value too large exception.
Dove vanno intercettate le eccezioni il prima possibile!
In generale le eccezioni vanno fatte subito, non mettendo un blocco grossissimo di funzione all'interno di un blocco tryexcept. I blocchi devono essere il più piccoli e specifici possibili.

\texttt{split()} per stringa usata per separare cose.

Ha fatto l'esempio di vario codice, e non torna! Bisogna imparare a capire dove cercare gli errori.

Funzione che fa parsing di una singola linea. ! \textbf{si può iterare sui file}!

Qua non ho nemmeno gestito l'errore, e vorrei invece sapere nel file quale linea è problematica.

\begin{minted}{python}
def parse_line(line):
	""" Parse a line of the file and return the values as float"""
	values = line.strip('\n').split(' ')
	# the following two lines may generate exceptions if they fail!
	time = float(values[0])
	tension = float(values[1])
	return time, tension

with open('snippets/data/fake_measurements.txt') as lab_data_file:
	for line in lab_data_file:
		if not line.startswith('#'): # skip comments
			time, tension = parse_line(line)
			print(time, tension)
\end{minted}

Qua invece sto prendendo l'errore troppo presto, non passo info interessanti.

\begin{minted}{python}
def parse_line(line):
	""" Parse a line of the file and return the values as float"""
	values = line.strip('\n').split(' ')
	try:
		time = float(values[0])
		tension = float(values[1])
	except ValueError as e:
		print(e) # This is not useful - which line of the file has the error?
		return None # We can't really return something meaningful
	return time, tension

with open('snippets/data/fake_measurements.txt') as lab_data_file:
	for line in lab_data_file:
		if not line.startswith('#'): # skip comments
			time, tension = parse_line(line)
			print(time, tension) # This line still crash badly!
\end{minted}

In questo caso è quello giusto da fare, così capisco anche la linea.

\begin{minted}{python}
def parse_line(line):
	""" Parse a line of the file and return the values as float"""
	values = line.strip('\n').split(' ')
	time = float(values[0])
	tension = float(values[1])
	return time, tension

with open('snippets/data/fake_measurements.txt') as lab_data_file:
	for line_number, line in enumerate(lab_data_file): # get the line number
	if not line.startswith('#'): # skip comments
		try:
			time, tension = parse_line(line)
			print(time, tension)
		except ValueError as e:
			print('Line {} error: {}'.format(line_number, e))
\end{minted}

Consiglio su cosa va usato. Quantità di codice scritto in Fortran e C è enorme e usati ovunque, quindi c'è valore nel saper programmare in queste cose. Se cominci ora da zero a scrivere qualcosa, scrivilo in C++ o Python, ma saper programmare in Fortran ha un valore solo per il materiale che c'è in giro.
Python puro è lento, ma ci sono molte librerie scritte in linguaggi compilati che consentono di fare cose velocemente. Ci sono cose che come tempo contano 0 e però ha tempo di sviluppo di molto.

\begin{minted}{python}
from splrand.pdf import ProbabilityDensityFuncion
import numpy as np

x = np.linspace(0., 1., 100)
y = 2. * x
pdf = ProbabilityDensityFunction(x, y)

# ora non chiamo il costruttore, ma un oggetto creato, e che ha un metodo chiamato call
pdf(0.5)
> [Output]: array(1.)
\end{minted}

Ad esempio, supponiamo di fare una classe, 

\begin{minted}{python}
class Dummy:

	def __init__(self, name):
		self.name = name

if __name__ == '__main__':
	d = Dummy('ciao')
	print(d.name)
	print(d(3))
	# che succede? type error, perché non è chiamabile!
\end{minted}

Questo metodo magico chiamato \mintinline{python}{__call__} è l'unico che consente di chiamare le cose, se lo aggiungo come 

\begin{minted}{python}
def __call__(self, value):
	return value * 3.
\end{minted}

Pagina sul data model di python, fra le varie cose abbiamo le basic customization, metodi magici che mimicano un certo numero di comportamenti dei tipi standard di python.
Chiamare il costruttore della classe è una cosa, che restituisce l'istanza di una classe; a quel punto dipende se l'oggetto ha un metodo per essere chiamato.
Sulla documentazione dice cosa succede per ogni metodo. Se non fai overload, il metodo della classe genitore viene utilizzato.
Quando invece di fare i loop sulle funzioni, tutte fatte nella funzione max di python, basta dire come due oggetti si confrontano, molto più facile da leggere e meno difficile da sbagliare.
Anche il \href{https://docs.python.org/3/reference/datamodel.html}{datamodel} da avere come la Bibbia.

\section{Lezione 8, lunedì 17 ottobre 2022}

\subsection{Iteratori}

Parlato di come si fa a rendere un oggetto iterabile alla python, che fa fare il ciclo for. Come si fa? si usa il metodo magico \mintinline{python}{__iter__}. Noi l'avevamo fatto riciclando il metodo \texttt{iter} della libreria \texttt{array}.
Un iteratore ha implementato il metodo \mintinline{python}{__next__}. In pratica si sa qual'è il prossimo elemento da restituire. Chiamando l'oggetto, vi restituisce quello successivo. Quando è finita l'iterazione, deve sollevare un'eccezione chiamata \texttt{StopIteration()}.
Perché devo implementare \texttt{iter} che usa \texttt{next} e devo implementare \texttt{next}? Perché così posso avere più iteratori attivi sullo stesso tipo di oggetto iterabile.
Motivo per cui un iteratore è tecnicamente un iterabile, ma non il viceversa è sempre vero!
Ora fa un esempio per capire che il ciclo for è equivalente a un ciclo infinito di chiamate fino a quando non esce l'eccezione:

\begin{minted}{python}
my_list = [1., 2., 3.]

# For-loop syntax
for element in my_list:
	print(element)

# This is equivalent (but much less readible and compact)
list_iterator = iter(my_list)
while True:
	try:
		print(next(list_iterator))
	except StopIteration:
		break
\end{minted}

Notiamo che se implementiamo \texttt{getitem} e \texttt{len} l'interprete si fa da sé l'iteratore.

Il seguente metodo funziona per le liste e non per i dizionari perché non solleva il \texttt{KeyError}.

\begin{minted}{python}
class SimpleIterator:
	""" Class implementing a super naive iterator"""
	def __init__(self, container):
		self._container = container
		self.index = 0

	def __next__(self):
		try:
			# Note: here we are calling the __getitem__ method of self._container
			item = self._container[self.index]
		except IndexError:
			raise StopIteration
		self.index += 1
		return item
	
	def __iter__(self):
		return self

class SimpleIterable:
	""" A very basic iterable """
	def __init__(self, *elements):
		# We use a list to store elements internally.
		# This provide us with the __getitem__ function
		self._elements = list(elements)

	def __iter__(self):
	return SimpleIterator(self._elements)
\end{minted}

Questo è un generatore basico. In generale posso fare iteratori a caso, tipo in \mintinline{python}{crazy_iterator} lo fa diverso e strano.

Esistono metodi che dato un iterabile ritornano un singolo valore.

Iteratori, sono utili da usare wrappando i container.
Infine, gli iteratori operano su dati esistenti!

\subsection{Generators}

Qualcosa che ad ogni interazione calcola il prossimo, per risparmiare memoria. Per calcolare qualcosa un attimo prima di usarla e non prima. 
Un minimo di procrastinazione.
Possiamo iterare su elementi che non esistono prima ma vengono calcolati mano mano. 
Ci sono due modi, o con le \textit{generator expressions}, o con le \textit{generator functions}. In realtà nella vita vera le generatrici li dà python, ad esempio \texttt{range()} in \texttt{python3}.
Una volta che abbiamo il generatore, possiamo usarlo per il for loop.

\begin{minted}{python}
for i in range(4): # generators act like iterators in for loop
	print(i)

data = [12, -1, 5]
square_data_generator = (x**2 for x in data) # generator expression!
# qua la differenza rispetto alle [] è che non viene calcolata o creata la lista, ma calcolato tutto solo dentro un loop.
for square_datum in square_data_generator: # again, works like an iterator
	print(square_datum)
\end{minted}

Siccome la lista nell'esempio di sopra non è calcolata per intero prima, se modifico `data` in corso di loop, anche la generazione segue le modifiche della lista, dato che non è calcolata prima.

la \textbf{funzione generatrice} è una funzione che si crea e dopo un po' c'è uno \texttt{yield}. Il risultato di quando lo creo è che viene restituito un generatore. 
Il generatore è un oggetto che ha il metodo \texttt{next}. Che fa? esegue tutto il codice fino a \texttt{yield}, dove ritorna il valore che segue \texttt{yield}. Quando finisce solleva \texttt{StopIteration}.
Di solito i generatori sono scritti con un loop all'interno.

\begin{minted}{python}
# Generator function that provides infinte fibonacci numbers
def fibonacci():
	a, b = 0, 1
	while True:
		yield a
		a, b = b, a + b

# We need to impose a stop condition externally to use it
max_n = 7
fib_numbers = []
for i, fib in enumerate(fibonacci()):
	if i >= max_n:
		break
	else:
		fib_numbers.append(fib)
print(fib_numbers)

# Another way of doing that is using 'islice' from itertools
import itertools
# Generator expression
fib_gen = (fib for fib in itertools.islice(fibonacci(), max_n))
print(list(fib_gen))
\end{minted}

Qua la funzione \texttt{fibonacci()} non restituisce nulla, ma viene creato un oggetto tipo generatore e restituito. Quando poi lo uso, metto una condizione di stop esterna.
Modo elegante è \texttt{.islice()} dalla libreria \texttt{itertools}. In pratica, crea con un generatore e un massimo, restituisce un iterabile. In generale, comunque, va notato come \mintinline{python}{fib_gen} non ho calcolato nulla, perché ho le parentesi tonde e non quadre.

Il generatore serve quando voglio generare elementi in maniera lazy.
Le funzioni generatrici di python sono davvero tantissime, la pagina 43 delle slides \texttt{lecture\_advanced\_2} ne elenca alcune e poi fa qualche esempio.

Un metodo carino è il \texttt{groupby}, in generale si usa e c'è anche in \texttt{pandas}.

\subsection{Lambda functions}

\textbf{Anonimous Functions} are a construct typical of functional programming, tipo \textit{Lambda calculus}.
In Python, crea una funzione in corsa senza dargli un nome. Sono limitate a espressioni singole, ritornate all'utente. Molti degli usi tipici della lambda function sono fatte dalle espressioni generatrici e \textit{list comprehension}.

la sintassi è 
\begin{minted}{python}
multiply = lambda x, y: x * y
\end{minted}

Questa è una funzione a una riga, e quindi mi serve una funzione al volo e non la devo definire prima.

\texttt{map} applica una funzione ad un iterabile.
In generale il loro use case è coperto dal linguaggio normale.
L'uso delle \texttt{lambda} è solo quando la funzione dopo è molto breve.

L'esercizio di ricapitolazione finale aveva senso ma non è molto illuminante, dice che preferisce esercitazione, lui che scrive codice.

Ora facciamo assignment advanced 2.

\subsection{Assignment Advanced 2}

Questo è stato buona parte trattato nel file dell'assegnamento, con qualche aggiunta sul file in Markdown.

\section{Advanced Python Features, the end}

\subsection{Funzioni}

Le funzioni sono oggetti di prima classe, significa che possono essere passati come argomenti ad altre funzioni e tornate come risultati da altre funzioni.
Del resto le funzioni sono oggetti di una classe. E si può definire un'altra funzione dentro una funzione.

Perché definire una funzione dentro una funzione? Ha senso, per esempio se devo tornarla come risultato della funzione.

Quando una funzione è creata all'interno di un'altra funzione ha accesso alle variabili locali della funzione esterna, anche se il suo scopo è esaurito. Questo è possibile perché le variabili sono tenute in uno speciale spazio di memoria, la \textbf{chiusura} della funzione interna.
Queste variabili sono chiamate variabili libere: in questo modo, una funzione può mantenere uno stato dentro la sua chiusura. Questo consente di usare le funzioni per compiti che in altri linguaggi sono riservati alle classi.

Ad esempio: 

\begin{minted}{python}
import numpy

def rotate(x, y, theta):
	""" Naive implementation. This works, but is inefficient - we have to
	calculate cos(theta) and sin(theta) for each pair (x,y)! """
	c = numpy.cos(theta)
	s = numpy.sin(theta)
	x_rot = c * x - s * y
	y_rot = s * x + c * y
	return x_rot, y_rot

x, y = 1., 0.
theta = numpy.pi/4
print(rotate(x, y, theta)) # Rotation of pi/2

def efficient_rotate(x, y, c_theta, s_theta):
	""" Efficient rotation. The user gives cos and sin, so they are not
	calculated for each pixel """
	x_rot = c * x - s * y
	y_rot = s * x + c * y
	return x_rot, y_rot

c = numpy.cos(theta)
s = numpy.sin(theta)
print(efficient_rotate(x, y, c, s)) # The syntax is very ugly!

def create_rotator(theta):
	""" Efficient rotation. Cosinus and sinus values are saved in the closure,
	so that they are computed exactly once."""
	c = numpy.cos(theta)
	s = numpy.sin(theta)
	def rotate(x, y):
		x_rot = c * x - s * y
		y_rot = s * x + c * y
		return x_rot, y_rot
	return rotate

x, y = 1., 0.
theta = numpy.pi/4
rotate_by_theta = create_rotator(theta)
print(rotate_by_theta(x, y))
\end{minted}

Si noti come la cosa più sensata è l'ultima ed è anche la più efficiente dal punto di vista del tempo computazionale.

Una cosa a cui fare attenzione è che definendo una variabile libera dentro una funzione interna, in automatico viene creata una variabile locale; se voglio invece usarne una non locale, devo usare l'indicatore \texttt{nonlocal}. \textit{Explicit is better than implicit}.

Invece di 
\begin{minted}{python}
def running_average():
	total_count = 0
	num_elements = 0
	def accumulator(value):
		total_count += value # Doesn’t work! total_count is reassigned!
		num_elements += 1 # Doesn’t work! total_count is reassigned!
		return total_count/num_elements
	return accumulator
\end{minted}

uno dovrebbe dichiararli non locali:

\begin{minted}{python}
def running_average():
	total_count = 0
	num_elements = 0
	def accumulator(value):
		# We declare the relevant variables as nonlocal
		nonlocal total_count, num_elements
		# Now we can assign to them - the variables in the closure will be
		# modified, as we want!
		total_count += value
		num_elements += 1
		return total_count/num_elements
	return accumulator
\end{minted}

In genere, si usano funzioni dentro funzioni solo quando si vuole fare un \textit{wrapper}.

\paragraph{Wrapper.} Un \textit{wrapper} è una funzione che chiama un'altra funzione aggiungendo un layer di funzionalità in mezzo, per esempio processare l'input o cambiare l'output in qualche modo formattato bene tipo. La tecnica per costruirlo è:
\begin{itemize}
	\item passare la funzione da wrappare come argomento della funzione esterna;
	\item dentro la funzione esterna definiamo una funzone interna, che è il wrapper;
	\item il wrapper chiama la funzione passata e aggiunge le funzioni;
	\item ritorniamo la funzione esterna.
\end{itemize}

Per fare un esempio sensato:
\begin{minted}{python}
def some_function(a, b):
	print(’Executing {} x {}’.format(a, b))
	return a * b

def add_n_wrapper(func, n): # We take the wrapped function as argument
	""" This wrapper adds n to the result of the wrapped function"""
	def wrapper(*args, **kwargs):
		"""We passs the arguments as *arg, **kwargs, 
		because this is the most general form in Python:
		we can collect any combination of arguments like that.
		Note that we have access to both ’func’ and ’n’,
		as they are stored in the closure of ’wrapper’"""
		result = func(*args, **kwargs) # Pass the arguments to the wrapped fucntion
		print(’Adding {}’.format(n))
		return result + n # Return a modified result in this case
	return wrapper # From add_n_wrapper we return the wrapper
\end{minted}

\subsection{Decoratori}

Per riassegnare la funzione wrappata al suo vecchio nome mantenendo le proprietà nuove, si usa una sintassi chiamata decoratori.
Un \textbf{decoratore} ha semplicemente il nome del \textit{wrapper} con un \textbf{\@} in cima alla sua dichiarazione.

\begin{minted}{python}
def print_function_info(func):
	def wrapper(*args, **kwargs):
		print('Calling function \'{}\''.format(func.__name__))
		print('Positional arguments = {}'.format(args))
		print('Keyword arguments = {}'.format(kwargs))
		return func(*args, **kwargs)
	return wrapper

@print_function_info
def some_function(a, b, c=0):
	return a * b + c

# This is equivalent to: some_function = print_function_info(some_function)

print(some_function(1, 2, c=7))
# Inspecting the function reveals that we are calling the wrapper
print('The name of the function is \'{}\''.format(some_function.__name__))
\end{minted}

Se ad esempio vogliamo fare un decoratore per il tempo,

\begin{minted}{python}
import time
from functools import wraps

def clocked(func):
	""" We use functools.wraps to keep the original function name and docstring"""
	@wraps(func)
	def wrapper(*args, **kwargs):
		tstart = time.clock()
		result = func(*args, **kwargs)
		exec_time = time.clock() - tstart
		print(’Function {} executed in {} s’.format(func.__name__, exec_time))
		return result
	return wrapper

@clocked
def square_list(input_list):
	""" Return the square of a list"""
	return [item**2 for item in input_list]

# Make sure the function name and docstring look the same
print(’\’{}\’: {}’.format(square_list.__name__, square_list.__doc__))
square_list(range(2000000))
\end{minted}


\paragraph{Metodi della classe.} Un decoratore particolare è il \mintinline{python}{@classmethod}. È tipo un attributo, non ti serve un'istanza per usarlo: un \textit{classmethod} può accedere agli attributi della classe ma non agli attributi dell'istanza. Serve per creare un costruttore alternativo.
Ad esempio:

\begin{minted}{python}
import numpy

class LabData:

	def __init__(self, times, values):
	""" Our usual constructor"""
		self.times = numpy.array(times, dtype=numpy.float64)
		self.values = numpy.array(values, dtype=numpy.float64)
	
	@classmethod # The classmethod decorator
	def from_file(cls, file_path): # We get the class as first argument, not self
		""" Constructor from a file"""
		print(cls)
		times, values = numpy.loadtxt(file_path, unpack=True)
		# We call the constructor of ’cls’ which is our LabData
		# This is not a ’real’ constructor, we need to return the object!
		return cls(times, values)

# We call the alternate constructor from the class itself, not from an instance!
lab_data = LabData.from_file(’snippets/data/measurements.txt’)
print(lab_data.values)
\end{minted}

\paragraph{Metodo Statico.} Lo \mintinline{python}{staticmethod} è un metodo che non riceve la classe o l'istanza come primo argomento; per questo motivo non altera lo stato della classe. In qualche senso un metodo statico è accoppiato alla classe in maniera vaga, è definito nella classe per convenienza ma potrebbe essere definito anche altrove.
Ad esempio:

\begin{minted}{python}
import math
class Angle:
	# A bunch of useful methods....
	@staticmethod
	def rad2deg(rad):
		# No self argument here
		return rad * 180./math.pi

	@staticmethod
	def deg2rad(deg):
		# No self argument here
		return deg * math.pi / 180.

print(Angle.rad2deg(math.pi/2))
print(Angle.deg2rad(45.))
\end{minted}

\paragraph{Passare argomenti ad un decoratore.} Per creare un decoratore che accetti argomenti, bisogna aggiungere un altro livello: la funzione \textit{decorator factory}. In pratica si hanno tre livelli:
\begin{itemize}
	\item la \textit{decorator factory} che prende i parametri per il decoratore e produce una funzione;
	\item il decoratore che prende come input una funzione e ritorna il \textit{wrapper};
	\item il \textit{wrapper} che implementa le funzionalità aggiuntive; di solito prende come input gli stessi parametri del decoratore e del \textit{wrapper}, e ritorna il suo risultato.
\end{itemize}
Per essere più chiari, questo è come dovrebbe funzionare:
\begin{minted}{python}
from functools import wraps

# This is how a generic decorator with arguments looks like

def decorator_factory(*params): # The signature can be anything
	def decorator(function):
		@wraps(function)
		def wrapper(*args, **kwargs):
			print(f’Decorator arguments: {params}’)
			# some preprocess here
			result = function(*args, **kwargs)
			# some post-process here
			return result
		return wrapper
	return decorator

# usage
@decorator_factory(1, 2, 3)
def f(x):
	return x**2

f(5)
\end{minted}

Gli effetti si possono incatenare, ad esempio:

\begin{minted}{python}
from decorator_samples import clocked, repeat, print_function_info

""" To chain the effect of more than one decorator just stack them above the
function definition """

@print_function_info
@repeat(num_times=3)
@clocked
def greet(name):
	print(f’Hello {name}’)

greet(’Bob’)
\end{minted}

\subsection{Ottimizzazione}

Per ottimizzazione si intende rendere il codice performante, mentre per \textit{profiling} si intende lo studio dei costi di risorse del tuo codice come funzione del tempo, dell'input e dello stato di esecuzione.
Il messaggio da portare a casa è che \textbf{l'ottimizzazione prematura è la radice di ogni male}.
Il corretto modo di fare codice:
\begin{itemize}
	\item scrivere codice corretto, leggibile, debuggato, testato e documentato;
	\item ignorare le performance
	\item se le performance sono buone, amen; altrimenti profila il codice per trovare dove performa peggio, e a quel punto inizia ad ottimizzare dal problema;
	\item fai test fino a che non hai una nuova versione.
\end{itemize}

Per capire cosa dovere ottimizzare, devi sapere più o meno come è fatto un computer. 
I dati sono processati dalla CPU o dalla GPU. La CPU ha alcune memorie cache, chiamate L1, L2 etc da pochi kB. 
L1 è sul singolo processore, le altre sono condivise fra vari processori. 
La connessione fra la cache e la CPU si chiama \textit{bus backside} ed è molto veloce. 
I dati poi vanno trasferiti dalla RAM alla cache, che avviene grazie al \textit{frontside bus} che è molto più lento del \textit{backend bus}.
Non solo, è anche più lento se devi recuperare i dati dalla memoria solida.

Ovviamente, per ottimizzare la cosa principale sarebbe fare meno operazioni possibili e quindi usare l'algoritmo ottimale per fare il lavoro.
Vanno ridotte al massimo le operazioni sulla memoria solida e sulla rete; assicurati che i dati richiesti dalla CPU siano già in cache e non devono essere recuperati dalla RAM.
Usare la vettorizzazione è sempre meglio, perché fa fare alla CPU molte più operazioni in contemporanea.
Se il sistema ha più di un processore, prova a mettere il lavoro in parallelo in più possibile.

Cosa va evitato? La CPU usa algoritmi di \textit{branch prediction} e \textit{pipelining} per caricare le istruzioni successive e i dati mentre vengono eseguiti quelli correnti.
Se questo processo fallisce, si ottengono errori di \textit{branch/cache misses} e un numero di \textit{stalled cycles} nel trasporto dati.
Inoltre, si dovrebbe cercare di tenere i dati in memoria contigui, così da velocizzare i trasferimenti: questo è perché una lista è peggio di un array. 
Infine, le allocazioni di memoria sono costose come risorse: il programma si ferma e aspetta che il sistema operativo trovi una locazione di memoria.
Sulle \textit{context switches} e \textit{cpu migrations} non si può fare molto, sono gestite dal sistema operativo.

Gli strumenti utilizzati in generale per fare ottimizzazione sarebbe:
{\itshape
\begin{itemize}
	\item time profiling:
	\begin{itemize}
		\item simple \texttt{print} statements;
		\item a time measuring decorator;
		\item the \texttt{timeit} Python module;
		\item the Unix \texttt{time} utility;
		\item \texttt{cProfile};
		\item \texttt{line\_profiler};
	\end{itemize}
	\item Bytecode study: \texttt{dis};
	\item CPU efficiency: \texttt{perf};
	\item Memory profiling: \texttt{heapy}.
\end{itemize}
}

\textbf{ATTENZIONE!} Qua fa un esercizio con il \emph{Julia set}, tratto dal libro \textit{High Performance Python}, pubblicato per O'Reilly.