\documentclass[10pt, a4paper, twosided, titlepage, draft]{book}

\usepackage{luca}
\title{Computing Methods for Experimental Physics and Data Analysis}
\author{Luca F. D'Alessandro}
\date{\today}

\begin{document}

\maketitle

\tableofcontents

\chapter{Python Basics}

\section{Lezione 2, giovedì 23 settembre 2022}

\emph{The Zen of Python} sembra una cosa strana però dà l'idea di buone pratiche di programmazione.
Ora fa le convenzioni per scrivere codice ordinato.
\url{https://peps.python.org/pep-0008/}:
questo link serve come linea guita ma è tanto dettagliato, anche troppo non ti serve tutto questo.
Sulle slides ci sono anche dei linters, ma vabbè.
Ok, ora ci sono le cose base delle variabili.

% TODO to be finished

\section{Lezione 3, giovedì 29 settembre 2022}



\section{Lezione 4, lunedì 3 ottobre 2022}



\section{Lezione 5, giovedì 6 ottobre 2022}



\section{Lezione 6, lunedì 10 ottobre 2022}



\section{Lezione 7, giovedì 13 ottobre 2022}



\section{Lezione 8, lunedì 17 ottobre 2022}


\chapter{Parallel Programming}



\chapter{Machine Learning}



\chapter{C++}



\end{document}